% This is a Dippa editor document template. The template suits best for academic papers
% written in Finnish.
%

% Add a document type 'article' that suits for academic papers
\documentclass[a4paper]{article}

% Encoding that fits to Finnish language
\usepackage[utf8]{inputenc}
\usepackage[T1]{fontenc}

% Hyphenation for Finnish
\usepackage[finnish]{babel}
\usepackage{verbatim}

% References.
% NOTICE: References are currently not supported by Dippa Editor
\usepackage{natbib}

\bibpunct{(}{)}{;}{a}{,}{,}

\usepackage{fullpage}

\usepackage[top=0pt, bottom=0pt, left=80pt, right=80pt]{geometry}

% Begin the document
\begin{document}

% Document title
\title{\huge Effective communication media for customer feedback in agile software projects}
\date{\vspace{-5ex}}
\maketitle

\large

\section{Introduction}
Agile software development methods empraise the importance of communication, especially rapid and intense feedback from customer.

\begin{comment}
Why is this research important? Is there a bigger phenomenon that this research of yours is part of? Why people in your profession should care about this thesis?
\end{comment}

\section{Literature}
In previous research it has been shown that the lack of communication is one of the biggest challenges faced by Agile teams. A lot of research has been made about communication in software projects. However, customer feedback has not been the main theme in these researches.

As a cure for lack of communication, face-to-face communication has been empraised in the Agile community and in the Agile manifesto. However, in the world of global business and multisite projects I believe that alternative communicatiomethods are needed.

I believe that different project phases require different communication tools. In addition, different communication contexts (e.g. requirements gathering, planning, estimation, feedback) require different communication tools. Thus, it makes sense to study not only communication in general but to dig deep in certain communication context, which in this paper is customer feedback.

\begin{comment}
What has been done related to this (mainly in academic publications)? What do the authors say about the topic? How does your research question relate to these previous studies? How do you apply them or add to them? Based on what they say, what do you say?
\end{comment}

\section{Research Question}

Can a visual feedback tool Hannotaatio enable rapid and intense feedback from customer to the development team?

\begin{comment}
Based on what other people have studied before, what is the question that no one has really answered yet? What is the main question, and what are perhaps the two or three sub-questions that you need to answer to be able to answer the main question? Be sure of what you write, because you will have to answer to this question ☺
\end{comment}

\section{Method}
In this paper Hannotaatio will be evaluated by different media and communication theories, such as Media Richness Theory and Media Synchronicity Theory. The theories are used to identify the properties of a communication medium that make a medium effective. 

Users of Hannotaatio will be introduced to gather real-life experiences about Hannotaatio and its suitability to Agile software projects.

\begin{comment}
How do you find an answer to the research question? How do you gather data? From where do you gather data? How do you analyze the data? Out of all the methods in the world, why did you choose this one? What is good about it and what is not? What were the alternative methods, and what were their pros and cons? 
\end{comment}

\section{Results}
According the media and communication theories, has some advantages compared to other communication media (e.g. face-to-face, phone, email). For example, the visuality of Hannotaatio provides a high level of naturalness.

The result of interviews was that Hannotaatio is promising tool, but it hasn't been widely used because x, x, x.

\begin{comment}
What is the answer to the research question? What are the answers to the sub-questions? Keep this simple and clear.
\end{comment}

\section{Discussion}
The users of Hannotaatio interviewed were all the receivers of the customer feedback i.e. developers. No feedback senders, i.e. customers, were interviewed.

\begin{comment}
How could someone criticize your results? Are they internally valid (the data was gathered and analyzed correctly)? Are the results externally valid (can they be generalized and how)? Based on the results, what can you say about the bigger picture you described in your introduction? How could someone apply your results for further research? Or perhaps apply in a non-research context (e.g., in a company or in everyday life)?
\end{comment}

\begin{comment}
KEEP IN MIND!
* A thesis is never read from the beginning to the end in a linear way: write each chapter as a “stand-alone”. 
* Make sure your research question, method, and results form a super clear and clean package: this is the question, this is what I did to find an answer, this is the answer. Your Mom should understand it.
* You will spend most of your sweat in the literature review.
* Choose an audience. Your professor is the most important (and perhaps the only) person who will read the thesis, but have also a wider audience in mind (e.g., colleagues, other professionals).
\end{comment}

% End of the document
\end{document}